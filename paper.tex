\documentclass{article}

\usepackage[hyphens]{url}
\usepackage{fancyvrb}
\usepackage{float}
\usepackage{amssymb}
\usepackage{amsmath}
\usepackage{amsfonts}
\usepackage{graphicx}
\usepackage{marvosym}

\newcommand\dataset{D}
\newcommand\datapoints{N}
\newcommand\datapoint{n}
\newcommand\datasubset[1]{\{ #1_\datapoint \}}
\newcommand\dimensions{M}
\newcommand\dimension{m}
\newcommand\catlabels{K}
\newcommand\catlabelindex{k}
\newcommand\binlabelindex{j}
\newcommand\realdata{{\bf x}}
\newcommand\realdataval{x}
\newcommand\allrealdata{X}
\newcommand\allcatlabels{C}
\newcommand\catlabel{c}
\newcommand\binlabel{b}
\newcommand\weights{{\bf w}}
\newcommand\weight{w}
\newcommand\labelerrprobs{{\bf z}}
\newcommand\labelerrprob{z}

\newcommand\logistic[1]{\sigma\left(#1\right)}

\newcommand\emobjective{{\cal E}}
\newcommand\emparams{\theta}
\newcommand\emiteration{i}
\newcommand\missing{B}
\newcommand\postprobvar{\beta}
\newcommand\postprobiter{\postprobvar^{(\emiteration)}}
\newcommand\postprob{\postprobiter_{\datapoint}}

\newcommand\reals{\mathbb{R}}
\newcommand\argmax{\mbox{argmax}}
\newcommand\transpose[1]{#1^T}

\newcommand\rcode[1]{{\tt #1}}


% Document
\begin{document}

%\frontmatter

\title{Logistic regression with a latent binary variable and noisy labels}
\author{Katrina Kalatar, Ian Holmes}
%\email{$\{$kkalantar,ihh$\}$@berkeley.edu}}

\maketitle

Following \cite{BootkrajangKaban2012},
consider a set of $\datapoints$ training data points
$\dataset = \left\{ (\realdata_1,\catlabel_1) \ldots (\realdata_\datapoints,\catlabel_\datapoints) \right\}$
where
$\realdata_\datapoint \in \reals^\dimensions$
denote $\dimensions$-dimensional real-valued data (e.g. gene expression levels)
and
$\catlabel_\datapoint \in \left\{ 0, 1 \ldots \catlabels - 1 \right\}$
denotes a categorical label with $\catlabels$ possible values
(e.g. clinician-assigned label incorporating some degree of uncertainty).

We aim to fit this with a two-stage model using a latent binary variable
$\binlabel_\datapoint \in \left\{ 0, 1 \right\}$
(e.g. true disease state) for each datapoint
\begin{eqnarray*}
P(\binlabel = 1 | \realdata, \weights) & = & \logistic{\weights^T \realdata} \\
P(\catlabel = \catlabelindex | \binlabel = \binlabelindex, \labelerrprobs) & = & \labelerrprob_{\binlabelindex\catlabelindex}
\end{eqnarray*}
where
$\logistic{x} = \frac{1}{1+e^{-x}}$ is the logistic function,
$\weights \in \reals^\dimensions$ are weight parameters for the logistic regression model, and
$\labelerrprobs$ are probability parameters for the label observation model.

We put a Laplace double-exponential (Lasso) prior on $\weights$, and a uniform Dirichlet prior on each row of $\labelerrprobs$
\begin{eqnarray*}
P(\weights) & \propto & \prod_{\dimension=1}^{\dimensions} \exp(-|\realdataval^{(\dimension)}|) \\
P(\labelerrprobs) & \propto & \prod_{\binlabelindex \in \{0,1\}} \delta \left( 1 - \sum_{\catlabelindex=0}^{\catlabels-1} \labelerrprob_{\binlabelindex\catlabelindex} \right)
\end{eqnarray*}

This is equivalent to Section 2.2 of \cite{BootkrajangKaban2012},
with the sum over $j$ in equation (8) of that paper constrained to $j\in\{0,1\}$
instead of $j\in\{0,1\ldots K-1\}$.
The paper derives a conjugate gradient optimization algorithm,
and proves its convergence.


\bibliographystyle{plain}
\bibliography{references}


\end{document}
